%%%%%%%%%%%%%%%%%%%%%%%%%%%%%%%%%%%%%%%%%%%%%%%%%%%%%%%%%%%%%%%%%%%%%%
%
% Section: Algebraic Substitution
%
%%%%%%%%%%%%%%%%%%%%%%%%%%%%%%%%%%%%%%%%%%%%%%%%%%%%%%%%%%%%%%%%%%%%%%

\section{Algebraic Substitution}
\label{AlgebraicSubstitution}

A 4000 year old Babylonian tablet, now on display at the Institute for the Study of the Ancient World at New York University, poses the following question:

\indenttext{\quotes{If the cost of digging a trench is 9 gin, and the trench has a length of 5 ninda and is one-half ninda deep, and if a worker’s daily load of earth costs 10 gin to move, and his daily wages are 6 se of silver, then how wide is the canal?} \footnote{Source: Exhibition Review: \quotes{Masters of Math, from Old Babylon} by Edward Rothstein. New York Times.  11/26/10}
} 

\bigskip

Except for the unfamiliar units (what the heck is a ninda or a gin or a se?) this looks like a pretty typical algebra problem, and you might imagine ancient Babylonian students sweating over this problem in algebra class. You might imagine them struggling through the wording to set up an equation, and then wrestling through the algebra to solve that equation and get the answer.  Except for one thing: algebra as we know it wasn’t invented until thousands of years later.  \footnote{There was no moment of invention, and no single inventor of algebra, so we can’t really pinpoint \quotes{when algebra was invented}. The methods we now call \quotes{algebra} developed over the course of more than 1000 years. But all of that happened long after the Babylonians. Most people consider the start of development of our familiar 	variables-and-equations algebra to the work of the Persian mathematician Muhammad ibn Mūsā al-Khwārizmī, who lived from around 780-850 CE. The word \quotes{algebra} comes from one of the words in the Arabic title of one of his books \quotes{Al-Kitāb al-mukhtaṣ ar fī ḥ isāb al-jabr wa-l-muqābala}.}

The methods available to Babylonian math students and their teachers didn’t include the tools of variables and equations that we are so used to today. They had their methods of course, but a problem like this was vastly more difficult to solve without the tools than we now take for granted.

Even though it may not always seem that way, the point of algebra is to make problem solving simpler. We use variables, algebraic expressions built from them, and equations using them to approach problem-solving in a logical and organized way. In this chapter, we’ll review some of the basic tools and methods of this approach.

%%%%%%%%%%%%%%%%%%%%%%%%%%%%%%%%%%%%%%%%%%%%%%%%%%%%%%%%%%%%
%
% Subsection: Algebraic Substitution: Substitution in Algebraic Expressions
%
%%%%%%%%%%%%%%%%%%%%%%%%%%%%%%%%%%%%%%%%%%%%%%%%%%%%%%%%%%%%

\subsection{Substitution in Algebraic Expressions}

In algebra we use symbols as a more compact alternative to ordinary language. For example, the fact:

\begin{center}
	The area of a rectangle is the result of multiplying its length by its width
\end{center}

can be expressed much more simply using mathematical symbols and variables in place of words and as the equation:

$$A=LW$$

Both the English sentence and the equation are saying the same thing, but it is usually advantageous to use symbols rather than words.

An \textbf{algebraic expression}\index{Algebraic Expression} is any collection of variables and/or numbers tied together by arithmetic operations. In the above equation, we used the algebraic expression \quotes{LW} to represent the instructions for calculating the area.

Now, suppose we have a rectangular garden that measures 20 feet by 16 feet. To calculate its area, we can use the equation above, plugging in 20 for $L$ and 16 for $W$, to get:

$$A = (20 \text{ feet})(16 \text{ feet}) = 320 \text{ square feet}$$

\quotes{Plugging in} specific values for the variables and evaluating the result is more formally known as \index{Algebraic Substitution} \textbf{algebraic substitution}.

\exam{\label{AlgebraicSubstitutionExample1} Evaluate the algebraic expression $t^2+2t-4p$ if $t=7$ and $p=-3$}

\indenttext{We start with the equation, and then replace the variables $t$ and $p$ with their given values, and then do the arithmetic:

$$t^2+2t-4p=(7)^2+2(7)-4(-3)=75$$

Since we don’t know what this algebraic expression is meant to represent or the appropriate units for the variables represent, we just leave the final answer as $75$.
}

\exam{\label{AlgebraicSubstitutionExample2} Evaluate $\frac{c^2V}{2}$ if $c=-3$ and $V=200$}

\indenttext{The original equation contains $c^2$, so whatever $c$ is, we need to square it. But if we write $-3^2$, remember that it is only the $3$ that gets squared. To make sure that the entire $-3$ will be squared, we need to use parentheses:
	
$$\frac{c^2 V}{2} = \frac{(-3)^2(200)}{2} = \frac{9*200}{2} = \frac{1800}{2} = 900$$
}

\bigskip

The example above warns of a common and easy-to-make mistake when substituting negative numbers. This error can always be avoided by putting parentheses around the number.  While parentheses are not always needed, you may want to consider always putting parentheses around negative numbers when you substitute them in, to play it safe.

\exam{\label{AlgebraicSubstitutionExample3} Evaluate $P^2T^3-1$ if $P=-4$ and $T=2$}

\indenttext{Substituting in the given values:
	$$P^2T^3-1=(-4)^2(2)^3-1=16*8-1=128-1=127$$
}

%%%%%%%%%%%%%%%%%%%%%%%%%%%%%%%%%%%%%%%%%%%%%%%%%%%%%%%%%%%%
%
% Subsection: Algebraic Substitution: Formulas
%
%%%%%%%%%%%%%%%%%%%%%%%%%%%%%%%%%%%%%%%%%%%%%%%%%%%%%%%%%%%%

\subsection{Formulas}

When we use an algebraic expression or an equation involving algebraic expressions to express a specific relationship between quantities, we usually call it a formula. In the discussion above, the equation $A=LW$ is the formula for the area of a rectangle. On the other hand, we would probably not call the algebraic expressions in Examples \ref{AlgebraicSubstitutionExample1}, \ref{AlgebraicSubstitutionExample2} and \ref{AlgebraicSubstitutionExample3} formulas, because we don’t really have any sense of those expressions’ purposes or meanings.

When you are just substituting into algebraic expressions without any sense of what they represent, all you can give in a final answer is the just numeric result. When you have know the units involved or the meaning of what you are calculating, it is appropriate to state your final answer using the appropriate units and/or in a way that relates it to the meaning.

\exam{\label{AlgebraicSubstitutionExample4}The formula for converting from temperature in Fahrenheit to temperature in Celsius is $C=\frac{5}{9}(F-32)$.  If the temperature is $14 ^\circ \text{F}$, what is the temperature in Celsius?
}

\indenttext{It is clear that in the formula $F$ represents temperature in degrees Fahrenheit and $C$ represents temperature in degrees Celsius. So we substitute given $F=14$ to get:

\begin{align*}
	C & =\frac{5}{9}(F-32)\\
	&\\
	& = \frac{5}{9}(14-32)\\
	&\\
	& =\frac{5}{9}(-18)\\
	&\\
	& = -10
\end{align*}

Since we know the units involved here, it would be most appropriate to state the final answer as \quotes{the temperature is $-10^\circ \text{C}$} or simply \quotes{$-10^\circ \text{C}$.}

\quotes{$C=-10$} would probably not be a satisfactory answer in this situation. If you asked someone \quotes{What is $14^\circ$ Fahrenheit in Celsius} and they answered \quotes{$C=-10$} you’d probably find that a pretty strange way of talking.
}

\bigskip

Of course, if you don’t know the appropriate units or the context, then you can only reasonably give the numeric answer.

\exam{\label{AlgebraicSubstitutionExample5}The formula for the energy stored in a charged capacitor is $W=\frac{1}{2}cv^2$.  Calculate $W$ if $c=100$ and $v=50$}

\indenttext{
\begin{align*}
	W & = \frac{1}{2}(100)(50)^2\\
	&\\
	& = \frac{1}{2}(100)(2500)\\
	&\\
	& = \frac{1}{2}(250,000)\\
	&\\
	& = 125,000
\end{align*}

In this case, the best we can do is say \quotes{the energy stored is $W=125,000$} or even just \quotes{$W=125,000$}, since we don’t know the units. If we don’t know the units or the meaning for a formula, obviously we can’t be expected to use them in the final answer. If, however, we did know the units, it would be appropriate to use them.
}

\exam{\label{AlgebraicSubstitutionExample6}The energy, in joules, stored in a charged capacitor is given by $W=\frac{1}{2}cv^2$, where $c$ is the capacitance in farads and $v$ is the current in volts. Determine the energy stored if the capacitance is 100 farads and the current is 50 volts.
}

\indenttext{This is the same problem as Example \ref{AlgebraicSubstitutionExample5}, except that here we know more about what the formula represents. A more appropriate answer to this question would be \quotes{the energy stored is 125,000 joules} or just \quotes{125,000 joules}.
}

%%%%%%%%%%%%%%%%%%%%%%%%%%%%%%%%%%%%%%%%%%%%%%%%%%%%%%%%%%%%
%
% Subsection: Algebraic Substitution: Input-Output and the Calculator
%
%%%%%%%%%%%%%%%%%%%%%%%%%%%%%%%%%%%%%%%%%%%%%%%%%%%%%%%%%%%%

\subsection{Input-Output and Your Calculator}

Sometimes we have a formula that we need to use repeatedly. Suppose that you are doing a presentation for a phys ed class where you want to show the relationship between daily exercise and weight. In your research, you’ve formulas that predict a person’s weight one year from now based on how much they exercise. For a 40 year old man who is 5’ 10” and weighs 200 lbs, and who consumes 2800 calories per day, his projected weight $W$ one year from now is given by the formula:

$$W=210-\frac{m}{3}$$

where $m$ is the number of minutes of vigorous exercise he gets on average per day.

You don’t think just showing this formula will give your audience a very good sense of the relationship between exercise and weight, and so you want to present a table, showing what his projected weight would be at different levels of exercise, something like this:

\begin{center}
\begin{tabular}{|c|c|}
	\hline
	If his average daily vigorous exercise is: & His projected weight in one year would be:\\
	\hline
	none &\\
	\hline
	30 &\\
	\hline
	60 &\\
	\hline
	90 &\\
	\hline
	120 &\\
	\hline
\end{tabular}
\end{center}

To fill in the projected weights in this table, you will need to substitute each of the different exercise amounts into the formula. This sort of table is called an input-output table, because it shows the result (output) you get with each different value you input into the formula.

It is not difficult to do this by hand, but it is a little tedious. Your TI-83 or TI-84 calculator has the built-in ability to create tables for a given formula. Let’s walk through how to use the calculator to create this table.

First, we need to give the calculator the formula we are using. In the upper left hand corner of your calculator, you should see a key labeled \quotes{Y=}. Hit this key now and you should see a screen that looks like this:

\begin{figure}[ht!]
	\centering
	\input{Sections/AlgebraicSubstitutionImages/CalculatorFormulaScreen.pdf_tex}
	\caption{Calculator Formula Input Screen}
\end{figure}

\FloatBarrier

If there are any formulas already entered, you can either type over them or delete them using the \quotes{CLEAR} key.

The calculator can keep track of multiple formulas at once, but we only need to use the one right now, so we will just enter our formula on the Y1 line.

While any letter at all can be used as a variable, the calculator is not quite so flexible. The input variable (the one you are going to substitute different values for) must always be X, and the output variable (the result) must always be Y. Since the \quotes{Y1=} is already built in, we just need to enter the \quotes{210-X/3} part of the formula. You can do that now using the calculator keypad. For the X, you’ll notice a key labeled \quotes{$\text{X},\text{T},\theta,\text{n}$}. Hitting that key will give you the X you need. The result should look like this:

\begin{figure}[ht!]
	\centering
	\input{Sections/AlgebraicSubstitutionImages/CalculatorFormulaEx6.pdf_tex}
	\caption{Formula Entered into Calculator}
\end{figure}

\FloatBarrier

Now, to set up the table. In the top row of your calculator you will see a key labeled \quotes{WINDOW} with \quotes{TBLSET} marked above it. Hit \quotes{2nd} and then that key, and you will arrive at the Table Setup screen, which should look like this (the specific values on your screen may differ)

\begin{figure}[ht!]
	\centering
	\input{Sections/AlgebraicSubstitutionImages/CalculatorTableSetupAsk.pdf_tex}
	\caption{Calculator Table Setup Screen}
\end{figure}

\FloatBarrier

For our table, the exercise amounts we are inputting start with 0 and go up in increments of 30 minutes. So, set the value for \quotes{TblStart} to 0, and \quotes{$\Delta$Tbl} to 30. The \quotes{$\Delta$} is the Greek letter delta, which is commonly used in math and science to represent \quotes{change} or \quotes{change in}. So we are telling the calculator to start the table at 0 minutes and then change that input value in increments of 30 minutes.

Next, set \quotes{Independent} to Auto. Independent variable is just another term for input variable; in this case we are telling the calculator to plug in values of that variable automatically based on our instructions. If you instead set this to \quotes{Ask} you will need to type in the input values you want to use manually. Dependent variable is another word for output variable, which we also want the calculator to find automatically, so we set that to Auto as well. The result should look like this:

\begin{figure}[ht!]
	\centering
	\input{Sections/AlgebraicSubstitutionImages/CalculatorTableSetupAuto.pdf_tex}
	\caption{Calculator Table Setup Screen}
\end{figure}

\FloatBarrier

Now we are ready to see our table. In the upper right hand corned there is a key labeled \quotes{GRAPH} and above it the marking \quotes{TABLE}. Hit \quotes{2nd} and then that key, and you should see this result:

\begin{figure}[ht!]
	\centering
	\input{Sections/AlgebraicSubstitutionImages/CalculatorTableEx6.pdf_tex}
	\caption{Calculator Automatically Generated Table}
\end{figure}

\FloatBarrier

Note that even though we only wanted to go as far as 120 minutes, the calculator keeps going. We never told it where to stop, and we never actually do. If you don’t care about the values past 120 minutes, just ignore them. (Likewise, if you want the calculator to go farther, you can continue the table in either direction by scrolling using the up and down arrow keys.)  And so, now we are ready to put it all together and make that presentation to our phys ed class:

\begin{center}
	\begin{tabular}{|c|c|}
		\hline
		If his average daily vigorous exercise is: & His projected weight in one year would be:\\
		\hline
		none & 210 lbs\\
		\hline
		30 & 200 lbs\\
		\hline
		60 & 190 lbs\\
		\hline
		90 & 180 lbs\\
		\hline
		120 & 170 lbs\\
		\hline
	\end{tabular}
\end{center}

In summary:

%%%%%%%%%%%%%%%%%%%%%%%%%%%%%%%%%%%%%%%%%%%%%%%%%%%%%%%%%%%%%%%%%%%%%%
%
% Definition: Tables on Calculator
%
%%%%%%%%%%%%%%%%%%%%%%%%%%%%%%%%%%%%%%%%%%%%%%%%%%%%%%%%%%%%%%%%%%%%%%

\begin{definition}
	\index{Calculator!Table}
	\textbf{\underline{Creating an Input-Output Table on a TI-83/84}}\\
	\bigskip
	\begin{enumerate}
		\item Hit the \quotes{Y=} key and enter the formula. Use Y1 (or Y2, Y3, etc. as needed) as the output variable, and change the input variable to X
		\item Use 2nd WINDOW (TblSet) to set up the table.
		\item Use TblStart and $\Delta$Tbl to set up the initial value of the input variable and increment, and set Indpnt to \quotes{Auto} to have the calculator set up the input values automatically.
		\item Ignore TblStart and $\Delta$Tbl (it doesn’t matter what they are set to) and set Indpnt to \quotes{Ask} to type in input values manually.
		\item Always set Dependent to \quotes{Auto}
		\item Hit 2nd GRAPH (Table) to see the table. If you set Indpnt to \quotes{Ask} type in the values you 	want to plug in in the X column.
	\end{enumerate}
\end{definition}

\exam{\label{AlgebraicSubstitutionExample7}Using the formula $C=\frac{5}{9}(F-32)$ given in Example \ref{AlgebraicSubstitutionExample4}, create a table on your calculator which shows the Celsius equivalent for temperatures of $0^\circ \text{F}$, $32^\circ \text{F}$, $68^\circ \text{F}$, $98.6^\circ \text{F}$, and $212^\circ \text{F}$.
}

\indenttext{First we enter the formula, replacing the input variable, using Y1 in place of C and X in pace of F.  We will put parentheses around the fraction 5/9 to make it clear that the (X-32) is not part of the denominator. (This is actually not needed in this case, since order of operations would evaluate it with only the 9 in the denominator, but it is still clearer with the extra parentheses.)
	
\begin{figure}[ht!]
	\centering
	\input{Sections/AlgebraicSubstitutionImages/CalculatorFormulaEx7.pdf_tex}
	\caption{Calculator Formula Entered on Formula Screen}
\end{figure}

\FloatBarrier

Now, since there is no set increment between the input values we want to use, we will set the Independent variable to Ask. (It does not matter what your TblStart and $\Delta$Tbl are set to; the calculator will ignore them.):

\begin{figure}[ht!]
	\centering
	\input{Sections/AlgebraicSubstitutionImages/CalculatorTableSetupEx7.pdf_tex}
	\caption{Calculator Table Setup Screen}
\end{figure}

\FloatBarrier

Now, we got to the table, and enter the input values we want. (If your calculator already has some X values in place you can just type over them.) The result should look like this:

\begin{figure}[ht!]
	\centering
	\input{Sections/AlgebraicSubstitutionImages/CalculatorTableEx7.pdf_tex}
	\caption{Calculator Table Generated from Entered Values}
\end{figure}
}

\FloatBarrier

%%%%%%%%%%%%%%%%%%%%%%%%%%%%%%%%%%%%%%%%%%%%%%%%%%%%%%%%%%%%%%%%%%%%%%
%
% Subsection: Algebraic Substitution: Tables with Multiple Inputs
%
%%%%%%%%%%%%%%%%%%%%%%%%%%%%%%%%%%%%%%%%%%%%%%%%%%%%%%%%%%%%%%%%%%%%%%

\subsection{Tables with Multiple Inputs}

The examples we’ve done here with tables both involved formulas where there was just one input variable. Could we create a table for a formula where we needed to input more than one value, such as the $W=\frac{1}{2}cv^2$ we used in Examples \ref{AlgebraicSubstitutionExample4} and \ref{AlgebraicSubstitutionExample5}?

The TI-83 and 84 calculators can do this, but they are really set up mainly for tables where there will only be one input value. In our work in this course, that will be fine. As we will see shortly, most of our attention in this course will be devoted to problems where there is just one input variable and one output variable. So we don’t need to concern ourselves at the moment with calculator tables for multiple inputs. Any time we need to substitute in multiple values, we will do this by substituting
and evaluating, as we did in Examples \ref{AlgebraicSubstitutionExample4} and \ref{AlgebraicSubstitutionExample5}.

%%%%%%%%%%%%%%%%%%%%%%%%%%%%%%%%%%%%%%%%%%%%%%%%%%%%%%%%%%%%%%%%%%%%%%
%
% Subsection: Algebraic Substitution: Exercises
%
%%%%%%%%%%%%%%%%%%%%%%%%%%%%%%%%%%%%%%%%%%%%%%%%%%%%%%%%%%%%%%%%%%%%%%

\clearpage

\subsection{Exercises}

\subsubsection*{Substituting into Algebraic Expressions}

Evaluate the given algebraic expressions by substituting in the given variable values.

\begin{tasks}[label={}](2)
	\task \ex{Evaluate $3x-1$ if $x=8$} \sol{23}	
	\task \ex{Evaluate $2y+5$ if $y=4$}
	\task \ex{Evaluate $2SP$ if $S=3$ and $P=7$} \sol{42}
	\task \ex{Evaluate $3k+m$ if $k=4$ and $m=5$}
	\task \ex{Evaluate $3z^2+3z+1$ if $z=-3$} \sol{19}
	\task \ex{Evaluate $2p+p^2$ if $p=3$}
	\task \ex{Evaluate $t^2+3t+1$ if $t=5$} \sol{41}
	\task \ex{Evaluate $2p+p^2$ if $p=-7$}
	\task \ex{Evaluate $w^2-20w+250$ if $w=-10$} \sol{550}
	\task \ex{Evaluate $2r^2-3r+5$ if $r=-2$}
	\task! \ex{Evaluate $\frac{T-R^2}{NR}$ if $T=100$, $R=4$, and $N=2$} \sol{10.5}
	\task! \ex{Evaluate $\frac{pq-12}{q}$ if $p=13$ and $q=2$}
\end{tasks}	


%%%%%%%%%%%%%%%%%%%%%%%%%%%%%%%%%%%%%%%%%%%%%%%%%%%%%%%%%%%%
%
% Exercises Formulas
%
%%%%%%%%%%%%%%%%%%%%%%%%%%%%%%%%%%%%%%%%%%%%%%%%%%%%%%%%%%%%

\subsubsection*{Formulas}

Use the given formulas and information to answer the questions posed. Phrase your answers appropriately for the information given. Round appropriately if necessary.

\ex{A publisher’s printing cost $C$ (in dollars) of producing $x$ copies of a textbook is given by the formula $C=400+25x$.  What would be the cost of producing 238 copies?} \sol{$ \$6350 $}

\bigskip
\ex{A printer’s cost $C$ (in dollars) to produce $x$ copies of a travel brochure is given by the formula $C=275+.24x$.  What would be the cost to print 3500 copies of the brochure?}

\bigskip
\ex{The height of a ball $h$ (in feet) $t$ seconds after it is thrown upward is given by the formula $h=-16t^2+60t+7$. What is the height of the ball three seconds after it is thrown?} \sol{$43$ feet}

\bigskip
\ex{The height of a javelin $h$ (in meters) $t$ seconds after it is thrown upward is given by the formula $h=-4.9t^2 + 17.4t+1.8$.  What is the javelin’s height four seconds after being thrown?}

\bigskip
\ex{At room temperature and pressure, the quantity of carbon dioxide $A$ (in moles) in a container having volume $V$ litres is given by the formula $A=\frac{V}{24}$.  What quantity of carbon dioxide will there be in a 6000 liter tank at room temperature and pressure?} \sol{$250$ moles}

\bigskip
\ex{At room temperature and pressure, the mass of carbon dioxide gas $M$ (in grams) in a container having volume $V$ liters is given by the formula $M=\frac{44V}{24}$. What is the mass of carbon dioxide contained in a 36 liter container at room temperature and pressure?}

\bigskip
\ex{The kinetic energy $E$ in joules possessed by an object with mass $m$ kilograms moving at $v$ meters per second is given by the formula $E=]frac{1}{2}mv^2$. Determine the kinetic energy of a 30 kilogram object moving at 12 meters per second.} \sol{$2160$ joules}

\bigskip
\ex{The kinetic energy $E$ in joules possessed by an object with mass $m$ kilograms moving at $v$ meters per second is given by the formula $E=\frac{1}{2}mv^2$. Determine the kinetic energy of a 15 kilogram object moving at 20 meters per second.}

\bigskip
\ex{A homeowner is considering installing a geothermal heating system. The system will cost a lot up front to install, but will cost much less to operate each year than her current system. The
time $T$ in years that it will take to recoup the extra cost of installing the system is given by the formula $T=\frac{12000}{K-300}$ where $K$ is the annual cost of operating her current system. How long will it take to recoup the cost if her current system costs $\$1800$ to operate?} \sol{$15$ years}

\bigskip
\ex{A retail store is considering replacing its current lighting fixtures with high-efficiency LED lighting. The LED lighting will cost more initially to install, but will cost less to operate. The time $T$ in months that it will take to recoup the additional cost of the LED lighting is given by the formula $T=\frac{4200}{D-170}$ where $D$ is the monthly cost of the current lighting. How long will it take to recoup the
LED system’s cost if the current lighting costs $\$380$ per month?}

\bigskip
\ex{A publisher’s average printing cost per textbook $a$ (in dollars) for a print run of $x$ copies of a textbook is given by the formula $a=\frac{400+25x}{x}$. What would be the average cost per textbook in a print run of 580 copies?} \sol{$\$25.69$}

\bigskip
\ex{A printer’s average cost per brochure $a$ (in dollars) to produce $x$ copies of a travel brochure is given by the formula $a=\frac{275+.24x}{x}$. What would be the cost to print 2500 copies of the brochure?}

%%%%%%%%%%%%%%%%%%%%%%%%%%%%%%%%%%%%%%%%%%%%%%%%%%%%%%%%%%%%%%%%%%%%%%
%
% Exercises Input-Output and a Calculator
%
%%%%%%%%%%%%%%%%%%%%%%%%%%%%%%%%%%%%%%%%%%%%%%%%%%%%%%%%%%%%%%%%%%%%%%

\subsubsection*{Input-Output Tables}

Create an Input-Output table in each of the following situations

\ex{A travel agent sells travelers checks denominated in foreign currencies to their customers.  The cost in US dollars of buying travelers checks worth $x$ euros is calculated using the formula $y=1.345x+27.50$. The agent wants to post a table showing what the cost would be to buy $\geneuro 250$, $\geneuro 500$, $\geneuro 750$, $\geneuro 1000$, $\geneuro 1250$ and $\geneuro 1500$ worth of travelers’ checks. Create the table he is looking for. (Note: \quotes{$\geneuro $} is the symbol for euros.)}
\sol{\begin{tabular}{c|c}
	$x$ & $y$ \\
	\hline
	$\geneuro 250$ & $\$363.75$ \\
	$\geneuro 500$ & $\$700.00$ \\
	$\geneuro 750$ & $\$1036.25$ \\
	$\geneuro 1000$ & $\$1372.50$ \\
	$\geneuro 1250$ & $\$1708.75$ \\
	$\geneuro 1500$ & $\$2045.00$ \\
\end{tabular}}

\bigskip
\ex{A flooring store wants to promote a sale in which they will install any amount of carpeting for just $\$75$ (plus the cost of the carpeting). During this sale, for their most popular carpet style the cost to carpet $x$ square feet can be found using the formula $y=3.98x+75$. The store manager wants to create a table which will show the total cost to carpet 300, 600, 900, 1200, 1500 and 1800 square feet. Create this table for him.}

\bigskip
\ex{The height of a ball $h$ (in feet) $t$ seconds after it is thrown upward is given by the formula $h=-16t^2+60t+7$. Create a table which shows the height of the ball 0, 0.5, 1.0, 1.5, 2.0, 2.5 and 3.0 seconds after being thrown.}
\sol{\begin{tabular}{c|c}
		$t$ & $h$ \\
		\hline
		$0.0$ seconds & $7$ feet \\
		$0.5$ seconds & $33$ feet \\
		$1.0$ seconds & $51$ feet \\
		$1.5$ seconds & $61$ feet \\
		$2.0$ seconds & $63$ feet \\
		$2.5$ seconds & $57$ feet \\
		$3.0$ seconds & $43$ feet \\
	\end{tabular}}

\bigskip
\ex{The distance traveled $d$ (in feet) by a bicycle $t$ seconds after hitting the brakes is given by $d=30t-2.5t^2$.  Create a table which shows the distance the bike has traveled 0, 0.5, 1.0, 1.5, 2.0, 2.5, 3.0, 3.5 and 4.0 seconds after hitting the brakes.}

\clearpage

%%%%%%%%%%%%%%%%%%%%%%%%%%%%%%%%%%%%%%%%%%%%%%%%%%%%%%%%%%%%%%%%%%%%%%