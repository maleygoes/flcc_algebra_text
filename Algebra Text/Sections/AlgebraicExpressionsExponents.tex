%%%%%%%%%%%%%%%%%%%%%%%%%%%%%%%%%%%%%%%%%%%%%%%%%%%%%%%%%%%%%%%%%%%%%%
%
% Section: Algebraic Expressions Exponents
%
%%%%%%%%%%%%%%%%%%%%%%%%%%%%%%%%%%%%%%%%%%%%%%%%%%%%%%%%%%%%%%%%%%%%%%

\section{Algebraic Expressions Exponents}
\label{AlgebraicExpressionsExponents}

%%%%%%%%%%%%%%%%%%%%%%%%%%%%%%%%%%%%%%%%%%%%%%%%%%%%%%%%%%%%%%%%%%%%%%
%
% Subsection: Algebraic Expressions Exponents: Multiplication
%
%%%%%%%%%%%%%%%%%%%%%%%%%%%%%%%%%%%%%%%%%%%%%%%%%%%%%%%%%%%%%%%%%%%%%%


\subsection{Exponents and Multiplication}

In an section \ref{AlgebraicExpressions}, we discovered the multiplication rule for exponents:

\begin{definition}
	\index{Exponents!Multiplication}
	\textbf{\underline{Multiplication Rule for Exponents}}\\
	\bigskip
	When groupings of the same variable raised to power or exponents, you can just add the exponents\\ (make sure to remember that a ‘plain’ $x$ is actually $x^1$).
	$$x^a \cdot x^b = x^{a+b}$$
\end{definition}

\exam{\label{ExponentsExample1}Rewrite the term $(6xy^3)(-3x^9)(4x^{10}y^4)$ in simplified form.}

\indenttext{Using the commutative and associative properties of multiplication, we can rewrite the expression as:

$$(6xy^3)(-3x^9)(4x^{10}y^4)=(6 \cdot -3 \cdot 4)(x^1x^9x^{10})(y^3y^4)$$

Multiplying the coefficients then adding the exponents of the x and y terms yields:

$$(6xy^3)(-3x^9)(4x^10y^4)=-72x^{1+9+10}y^{3+4}=-72x^{20}y^7$$
}

\exam{\label{ExponentsExample2}Rewrite the term $(-2c^3de^5)(10d^3e^7)(-3c^9d^5e^9)$ in simplified form.}

\indenttext{Rewriting we have:

$$(-2c^3de^5)(10d^3e^7)(-3c^9d^5e^9) = (-2 \cdot 10 \cdot -3)(c^3c^9)(dd^3d^5)(e^5e^7e^9)$$

Multiplying the coefficients then adding the exponents:

$$(-2c^3de^5)(10d^3e^7)(-3c^9d^5e^9)=(-60)(c^{3+9})(d^{1+3+5})(e^{5+7+9})=-60c^{12}d^{9}e^{21}$$
}

%%%%%%%%%%%%%%%%%%%%%%%%%%%%%%%%%%%%%%%%%%%%%%%%%%%%%%%%%%%%%%%%%%%%%%
%
% Subsection: Algebraic Expressions Exponents: Division
%
%%%%%%%%%%%%%%%%%%%%%%%%%%%%%%%%%%%%%%%%%%%%%%%%%%%%%%%%%%%%%%%%%%%%%%

\subsection{Exponents and Division}

That covers the multiplication rule. What happens if we have division of expressions with the same base (or the ratio of two expressions with the same base)? For example:

$$\frac{f^5}{f^2}$$

Expanding this expression yields:

$$\frac{f \cdot f \cdot f \cdot f \cdot f}{f \cdot f}$$

Since $\frac{f}{f}=1$, we can make use of this with 2 pairs (top and bottom) to produce:

$$\frac{f}{f} \cdot \frac{f}{f} \cdot f \cdot f \cdot f=1 \cdot 1 \cdot f \cdot f \cdot f=f \cdot f \cdot f=f^3$$

Notice that we divided out the $2$ factors of $f$ in the denominator from the $5$ factors of $f$ in the numerator leaving $3$ factors of $f$ in the numerator. Do you notice a connection between the $5$ and $2$ that will produce the $3$?

$$\frac{f^5}{f^2}=f^{5-3}=f^3$$

\begin{definition}
	\index{Exponents!Division}
	\textbf{\underline{Division Rule for Exponents}}\\
	\bigskip
	When dividing groupings of the same variable raised to powers, you can subtract the powers in the following manner:
	$$\frac{x^a}{x^b}=x^{a-b}$$
\end{definition}

\exam{\label{ExponentsExample3}Rewrite the term 

$$\frac{g^7h^4}{g^6h}$$ 

in simplified form.
}

\indenttext{Using the division rule for exponents, we have:

$$\frac{g^7h^4}{g^6h}=g^{7-6}h^{h^{4-1}}=gh^3$$
}

\exam{\label{ExponentsExample4}Rewrite the term 
$\displaystyle \frac{16p^{10}q^{13}}{2p^{4}q^{13}}$ 
in simplified form.
}

\indenttext{Using the division rule for exponents, we have:

$$\frac{16p^{10}q^{13}}{2p^{4}q^{13}}=\frac{16}{2}p^{10-4}q^{13-13}=8p^6q^0=8p^6$$
}

Note: In this example we used the fact that anything raised to the zero power is equal to 1.  This come from the observation that any nonzero quantity divided by itself is equal to 1.  In this example $q$ is divided by itself 13 times leading to the result that every factor of $q$ is reduced to one in the expression.

\exam{\label{ExponentsExample5}Rewrite the term $\displaystyle \frac{-6r^{8}s^{4}t^{18}}{-36r^{5}t^{12}}$
in simplified form.
}

\indenttext{Following the rules for division:
	
$$\frac{-6r^{8}s^{4}t^{18}}{-36r^{5}t^{12}}=\frac{-6}{-36}r^{8-5}s^{4}t^{18-12}=\frac{1}{6}r^3s^4t^6$$
}

\exam{\label{ExponentsExample6}Rewrite the term $\displaystyle \frac{24a^{10}b^{5}}{12a^{4}b^{5}}$
in simplified form.
}

\indenttext{We proceed:
	
$$\frac{24a^{10}b^{5}}{12a^{4}b^{5}}=\frac{24}{12}a^{10-4}b^{5-5}=2a^6$$
}

\exam{\label{ExponentsExample7}Rewrite the term 
$\displaystyle \frac{(2u^{9}v^{4})(5uv^{7})}{(10u^{3}v)(3u^{5}v^{7})}$
in simplified form.
}

\indenttext{If you are unsure whether to use the multiplication or division rule first, think of the expression:
	
$$\frac{6 \cdot 10}{4 \cdot 3}$$

You would probably multiply the terms of the numerator together, then multiply the terms of the denominator together, and finally divide the results to get:

$$\frac{6 \cdot 10}{4 \cdot 3} = \frac{60}{12}=5$$

We proceed with this example in the same manor:
\begin{align*}
	\frac{(2u^{9}v^{4})(5uv^{7})}{(10u^{3}v)(3u^{5}v^{7})} & = \frac{2 \cdot 5u^{9+1}v^{4+7}}{10 \cdot 3u^{3+5}v^{1+7}}\\
	& \\
	& = \frac{10u^{10}v^{11}}{30u^{8}v^{8}}\\
	& \\
	& = \frac{1}{3}u^{10-8}v^{11-8}\\
	& \\
	& = \frac{1}{3}u^2v^3
\end{align*}
}

%%%%%%%%%%%%%%%%%%%%%%%%%%%%%%%%%%%%%%%%%%%%%%%%%%%%%%%%%%%%%%%%%%%%%%
%
% Subsection: Algebraic Expressions Exponents: to Exponents
%
%%%%%%%%%%%%%%%%%%%%%%%%%%%%%%%%%%%%%%%%%%%%%%%%%%%%%%%%%%%%%%%%%%%%%%

\subsection{Exponents and Expressions}

What happens if we have a term with an exponent raised to an exponent? For example, $(x^4)^3$.

Expanding this as $(x^4)(x^4)(x^4)$, we can add the exponents $x^{4+4+4} = x^{12}$.

\exam{\label{ExponentsExample8}Rewrite the term $(a^2b^3)^4$ in simplified form.}

\indenttext{Expanding:
$$(a^2b^3)^4=(a^2b^3)(a^2b^3)(a^2b^3)(a^2b^3)$$

Regrouping and adding the exponents:
$$(a^2b^3)^4=(a^{2+2+2+2}b^{3+3+3+3})=a^8b^{16}$$
} 

Notice anything over the last two examples? When an exponent is raised to an exponent you can multiply the exponents. This is called the power rule for exponents.

\begin{definition}
	\index{Exponents!Power Rule}
	\textbf{\underline{Power Rule for Exponents}}\\
	\bigskip
	When raising groupings of the same variable raised to exponents, are then raised to a power or exponent, you can multiply the powers or exponents:
	$$\displaystyle {(x^a)}^b=x^{a \cdot b}$$
\end{definition}

\exam{\label{ExponentsExample9}Rewrite the term $\displaystyle {(d^{12}e^4f)}^9$ in simplified form.}

\indenttext{Using the power rule:
$${(d^{12}e^4f)}^9=d^{12\cdot 9}e^{4\cdot 9}f^{1\cdot 9}=d^{108}e^{39}f^9$$
}

What happens if the expression has a coefficient? For example:
$${(3g^2)}^3$$

Expanding:
$${(3g^2)}^3=(3g^2)(3g^2)(3g^2)$$

Regrouping:
$${(3g^2)}^3=(3)(3)(3)(g^2)(g^2)(g^2)$$

Multiplying coefficient and adding exponents:
$${(3g^2)}^3=27g^6$$

An alternate method is to remember that the coefficient of 3 has an exponent of 1, then use the power rule for exponents:
$${(3g^2)}^3={(3^1g^2)}^3=(3^{1\cdot 3}g^{2\cdot 3})=(3^{3}g^{6})=27g^6$$

\exam{\label{ExponentsExample10}Rewrite the term $\displaystyle {(-2h^3jk^5)}^5$ in simplified form.}

\indenttext{Applying the power rule:
\begin{align*}
	{(-2h^3jk^5)}^5 & = {(-2^1h^3j^1k^5)}^5 \\
	& \\
	& = (-2^{1\cdot 5}h^{3\cdot 5}j^{1\cdot 5}k^{5\cdot 5}) \\
	& \\
	& = -32h^{15}j^{5}k^{25} 
\end{align*}
}

%%%%%%%%%%%%%%%%%%%%%%%%%%%%%%%%%%%%%%%%%%%%%%%%%%%%%%%%%%%%%%%%%%%%%%
%
% Subsection: Algebraic Expressions Exponents: Fractions
%
%%%%%%%%%%%%%%%%%%%%%%%%%%%%%%%%%%%%%%%%%%%%%%%%%%%%%%%%%%%%%%%%%%%%%%

\subsection{Exponents and Fractions}

What happens if we have a rational expression raised to an exponent? For example:
$${\left(\frac{2}{5}\right)}^2$$

Entering this expression into your calculator as \quotes{(2/5)$^\wedge$2} yields 0.16. Converting this into a fraction gives us $4/25$. This demonstrates that we can apply the exponent of 2 to both bases (2 and 5) present in the numerator and denominator.  That is:
$${\left(\frac{2}{5}\right)}^2=\frac{2^2}{5^2}=\frac{4}{25}$$

\exam{\label{ExponentsExample11}Rewrite the term ${\displaystyle \left(\frac{m^5}{n^3}\right)}^4$ in simplified form.}

\indenttext{Applying the exponent to the numerator and denominator:
\begin{align*}
	{\left(\frac{m^5}{n^3}\right)}^4 & = \frac{{\left(m^5\right)}^4}{{\left(n^3\right)}^4}\\
	& \\
	& =\frac{m^{5 \cdot 4}}{n^{3\cdot 4}} \\
	& \\
	& = \frac{m^{20}}{n^{12}}
\end{align*}
}

\exam{\label{ExponentsExample12}Rewrite the expression $\displaystyle {\left(\frac{4g^7}{7h}\right)}^3$ in simplified form.}

\indenttext{Applying exponents rules:
\begin{align*}
	{\left(\frac{4g^7}{7h}\right)}^3 & = \frac{{\left(4g^7\right)}^3}{{\left(7h\right)}^3}\\
	& \\
	& =\frac{4^3{(g^7)}^3}{7^3h^3}\\
	& \\
	& = \frac{64g^{3\cdot7}}{343h^3}\\
	& \\
	& =\frac{64g^{21}}{343h^3}
\end{align*}
}

\exam{\label{ExponentsExample13}Rewrite the expression $\displaystyle {\left(\frac{2pq^5}{-3q^2}\right)}^4$ in simplified form.}

\indenttext{One approach is to simplify the expression inside the parenthesis using the division rule before regrouping:
\begin{align*}
	{\left(\frac{2pq^5}{-3q^2}\right)}^4 & = {\left(\frac{-2}{3}pq^{5-2}\right)}^4\\ 
		& \\
		& = {\left(\frac{-2}{3}\right)}^4p^4{\left(q^{3}\right)}^4\\
		& \\
		& = \frac{(-2)^4}{(3)^4}p^4q^{3 \cdot 4}\\
		& \\
		& = \frac{16}{81}p^4q^{12}
\end{align*}
}

%%%%%%%%%%%%%%%%%%%%%%%%%%%%%%%%%%%%%%%%%%%%%%%%%%%%%%%%%%%%%%%%%%%%%%
%
% Subsections: Algebraic Expressions Exponents: Exercises
%
%%%%%%%%%%%%%%%%%%%%%%%%%%%%%%%%%%%%%%%%%%%%%%%%%%%%%%%%%%%%%%%%%%%%%%

\clearpage

\subsection{Exercises}

Rewrite the following in simplest form

\begin{tasks}[label={}](2)
	\task\ex{$2g(5g)$} \sol{$10g^2$}
	\task\ex{$-3h(10h)(h^3)$}
	\task\ex{$(a^2b)(a^5b^3)$} \sol{$a^7b^4$}
	\task\ex{$(4xy)(3x^2y)$}
	\task\ex{$(5c^4d^7)(-2cd^9)$} \sol{$-10c^5d^16$}
	\task\ex{$(x^3y^4z^2)(x^5y^4z)$}
	\task\ex{$(6kj^8m^5)(13k^{20}j^3m^{10})$} \sol{$78k^{21}j^{11}m^{15}$}
	\task\ex{$(-2m^3p^{11})(-0.5p^7n^9)$}
	\task\ex{$(p^2q^3r)(r^5)(p^7q^{15}r^4)$} \sol{$p^{9}q^{18}r^{10}$} 
	\task\ex{$(5ef^4)(2f^7)(-2e^{12}f^7)$}
	\task\ex{$\left(15q^2rs^{15}\right)\left(\frac{1}{5}q^{17}r^{6}s^{20}\right)\left(-q^{7}r^{13}s\right)$} \sol{$3q^{26}r^{20}s^{36}$}
	\task\ex{$(3e^5f^6g^2)(12f^3g^{12})(-7e^4g)$}
	\task\ex{$\displaystyle \frac{2^5}{2^3}$} \sol{$4$}
	\task\ex{$\displaystyle \frac{3^7}{3^4}$}
	\task\ex{$\displaystyle \frac{h^8}{h^2}$} \sol{$h^6$}
	\task\ex{$\displaystyle \frac{15a^{20}}{3a^{15}}$}
	\task\ex{$\displaystyle \frac{4c^6}{20c^5}$} \sol{$\frac{1}{5}c$}
	\task\ex{$\displaystyle \frac{60g^2}{12h^{10}}$}
	\task\ex{$\displaystyle \frac{x^9y^{16}}{x^3y^4}$} \sol{$x^6y^{12}$}
	\task\ex{$\displaystyle \frac{108p^7z^{11}}{24pz^{10}}$} 
	\task\ex{$\displaystyle \frac{12y^{10}z^7}{14y^8z}$} \sol{$\frac{6}{7}y^2z^6$}
	\task\ex{$\displaystyle \frac{28f^5p^{10}}{-16fp^2}$} \task\ex{$\displaystyle \frac{-24a^4b^8c^9}{6ab^3c^3}$}  \sol{$-4a^3b^5c^6$}
	\task\ex{$\displaystyle \frac{60s^{23}t^{60}u^{100}}{84s^5tu^{99}}$} 
	\task\ex{$\displaystyle \frac{2^{1,355}}{2^{1,303}}$} \sol{$2^{52}$}
	\task\ex{$\displaystyle \frac{(a^4b^3)(a^5b^7)}{(a^2b)(a^6b^3)}$}
	\task\ex{$\displaystyle \frac{(xyz)(x^3y^4z^5)(x^7y^5z^{10})}{(x^{10}y^8z^9)}$}\sol{$xy^2z^7$}
	\task\ex{$\displaystyle \frac{(100m^{56})(30m^{11}b^{47})}{15m^{10}b^{25}}$}
	\task\ex{$\displaystyle \frac{(4c^9d^{10})(20c^{13}d^4)}{(5c^2d^5)(8c^7d^4)}$}\sol{$2c^{12}d^{5}$}
	\task\ex{$\displaystyle \frac{(8f^3g^9h^{15})(10f^{11}g^0h^9)}{(300f)(2f^{10}g^7h^4)}$}
	\task\ex{${(3^3)}^4$} \sol{$3^{12}$}
	\task\ex{${\left((-2)^3)\right)}^4$}
	\task\ex{${\left(-2^2\right)}^3$}\sol{$-2^6=-64$}
	\task\ex{${\left(2a^3\right)}^2$}
	\task\ex{${\left(-12b^7\right)}^3$}\sol{$-1728b^{21}$}
	\task\ex{${\left(hk^2\right)}^3$}
	\task\ex{${\left(3s^3v^4\right)}^5$}\sol{$243s^{15}{v^20}$}
	\task\ex{${\left(-2n^9o^{17}\right)}^4$}
	\task\ex{${\left(j^2l^4m\right)}^5$}\sol{$j^{10}l^{20}m^5$}
	\task\ex{${\left(-4qr^8s^{10}\right)}^3$}
	\task\ex{$\displaystyle{\left(\frac{1}{3}\right)}^4$}\sol{$\frac{1}{81}$}
	\task\ex{$\displaystyle{\left(\frac{2}{5}\right)}^3$}
	\task\ex{$\displaystyle{\left(\frac{g}{2}\right)}^2$}\sol{$\frac{g^2}{4}$}
	\task\ex{$\displaystyle{\left(\frac{1}{3}b^4\right)}^5$}
	\task\ex{$\displaystyle{\left(\frac{b^4}{3}\right)}^5$}\sol{$\frac{b^{20}}{243}$}
	\task\ex{$\displaystyle{\left(\frac{2t}{5v}\right)}^3$}
	\task\ex{$\displaystyle{\left(\frac{15x^5}{5x^2}\right)}^3$}\sol{$125x^9$}
	\task\ex{$\displaystyle{\left(\frac{w^2x}{y^5}\right)}^4$}
	\task\ex{$\displaystyle{\left(\frac{j^5km^{10}}{bc^8}\right)}^{40}$}\sol{$\frac{j^{200}k^{40}m^{400}}{b^{40}c^{320}}$}
	\task\ex{$\displaystyle{\left(\frac{\left(2xy\right)\left(3x^2y^5\right)}{\left(6xy^5\right)}\right)}^4$}
\end{tasks}

\clearpage

%%%%%%%%%%%%%%%%%%%%%%%%%%%%%%%%%%%%%%%%%%%%%%%%%%%%%%%%%%%%%%%%%%%%%%