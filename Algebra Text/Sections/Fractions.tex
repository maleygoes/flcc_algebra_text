
%%%%%%%%%%%%%%%%%%%%%%%%%%%%%%%%%%%%%%%%%%%%%%%%%%%%%%%%%%%%%%%%%%%%%%
%
% Section: Arithmetic with Fractions
%
%%%%%%%%%%%%%%%%%%%%%%%%%%%%%%%%%%%%%%%%%%%%%%%%%%%%%%%%%%%%%%%%%%%%%%

\section{Arithmetic with Fractions}
\label{ArithmeticwithFractions}

We often encounter fractions in mathematics and in the real world. Before moving forward in this book, we’ll take some time to review the rules of fraction arithmetic, and also discuss using your calculators for fractions.

%%%%%%%%%%%%%%%%%%%%%%%%%%%%%%%%%%%%%%%%%%%%%%%%%%%%%%%%%%%%%%%%%%%%%%
%
% Subsection: Arithmetic with Fractions: Adding and Subtracting Fractions
%
%%%%%%%%%%%%%%%%%%%%%%%%%%%%%%%%%%%%%%%%%%%%%%%%%%%%%%%%%%%%%%%%%%%%%%

\subsection{Adding and Subtracting Fractions}

Suppose you are traveling from New York to Singapore. You have a layover in Vancouver, where you pick up a book to read on the flight for C\$9.95 (Canadian dollars). Then, on your second layover in Tokyo you pick up a souvenir fridge magnet for ¥2500 (Japanese yen). How much have you spent?

Obviously you can’t just add 9.95 + 2500 to answer this question. Canadian dollars and Japanese yen are different currencies, and so you can’t just add them together and expect to get an answer that means anything. A sensible answer to the question requires you to put the amounts of money in some common terms, such as converting them to US dollars, and then adding the results.

The same issue arises if we need to add fractions. Halves and thirds are different things, and so to add we first need to put these fractions into common terms. We’ll recall the method for doing this below, and then work through a few examples.

%%%%%%%%%%%%%%%%%%%%%%%%%%%%%%%%%%%%%%%%%%%%%%%%%%%%%%%%%%%%%%%%%%%%%%
%
% Definition: Addition of Fractions
%
%%%%%%%%%%%%%%%%%%%%%%%%%%%%%%%%%%%%%%%%%%%%%%%%%%%%%%%%%%%%%%%%%%%%%%

\begin{definition}
	\index{Fractions!Addition}
	\textbf{\underline{Adding and Subtracting Fractions}}\\
	\bigskip
	\begin{itemize}[leftmargin=*]
		\item Find a common denominator.  A common denominator is any number that all of the denominators divide into evenly.
		\item A common denominator can always be found by multiplying all the denominators together.  Often, though, this will be larger than necessary.
		\item It is desirable, but not mandatory, to use the least common denominator. If the denominators share any factors in common, each factor only needs to be included in the	common denominator as many times as it is used in the denominator in which it appears the most.
		\item Each fraction can be changed to the common denominator by multiplying its numerator and denominator by whatever is needed to change the denominator into the common	denominator.
		\item Once all the denominators are common, the fractions can be added/subtracted by adding/subtracting their numerators.
	\end{itemize}
\end{definition}

Putting this in words makes it seem more complicated than it actually is. Some examples will help remind you how this all works.

\exam{\label{ArithmeticwithFractionsExample1}Evaluate $\displaystyle \frac{3}{5} + \frac{4}{3}$.}

\indenttext{The denominators are $3$ and $5$, and so a common denominator is $3 \cdot 5 = 15$. Since $3$ and $5$ don’t have any common factors, this is the smallest possible common denominator.

To make $5$ into the common denominator of $15$, we need to multiply by $3$. So we’ll multiply the first fraction by $\frac{3}{3}$ (recall that we can do this because $\frac{3}{3} = 1$, and so this changes the form of the fraction but not its value, since we are really just multiplying by $1$). For similar reasons, we’ll multiply the second fraction by $\frac{5}{5}$.

$$\displaystyle \frac{3}{5} \left( \frac{3}{3} \right) + \frac{4}{3} \left( \frac{5}{5} \right) = \frac{9}{15} + \frac{20}{15} = \frac{29}{15}$$

Note that this answer is an \textbf{improper fraction}, which means that its numerator is larger than its denominator. This answer could be rewritten as the \textbf{mixed number}; $1\frac{14}{15}$. Either answer is correct.  In this book, we will not rewrite improper fractions as mixed numbers, unless there is a clear reason to do so.
}

\exam{\label{ArithmeticwithFractionsExample2}Evaluate $\displaystyle \frac{3}{8} + \frac{5}{6} - \frac{2}{3}$}

\indenttext{We could use $8 \cdot 6 \cdot 3 = 142$ as a common denominator, but this is larger than necessary.  Since $8=2\cdot 2\cdot 2$ and $6=3\cdot 2$, and $3$ is just $3$, we only need to include three $2$’s (since $8$ has three of them) and one $3$ (since that one three will cover for both the $6$ and the $3$ denominators.) So our least common denominator is $2\cdot 2\cdot 2\cdot 3=24$. Then:
	
\begin{align*}
	\frac{3}{8} + \frac{5}{6} - \frac{2}{3} & =\frac{3}{8} \left( \frac{3}{3} \right) + \frac{5}{6} \left( \frac{4}{4} \right) - \frac{2}{3} \left( \frac{8}{8} \right)\\
	\\
	& = \frac{9}{24} + \frac{20}{24} - \frac{16}{24} \\
	& \\
	& = \frac{13}{24}
\end{align*}
}

\bigskip

%%%%%%%%%%%%%%%%%%%%%%%%%%%%%%%%%%%%%%%%%%%%%%%%%%%%%%%%%%%%%%%%%%%%%%
%
% Subsection: Arithmetic with Fractions: Multiplying and Dividing Fractions
%
%%%%%%%%%%%%%%%%%%%%%%%%%%%%%%%%%%%%%%%%%%%%%%%%%%%%%%%%%%%%%%%%%%%%%%

\subsection{Multiplying and Dividing Fractions}

Multiplying fractions is actually much easier than adding or subtracting them. No common denominators are required.

%%%%%%%%%%%%%%%%%%%%%%%%%%%%%%%%%%%%%%%%%%%%%%%%%%%%%%%%%%%%%%%%%%%%%%
%
% Definition: Multiplication of Fractions
%
%%%%%%%%%%%%%%%%%%%%%%%%%%%%%%%%%%%%%%%%%%%%%%%%%%%%%%%%%%%%%%%%%%%%%%

\begin{definition}
	\textbf{\index{Fractions!Multiplying}}
	\underline{Multiplying Fractions}\\
	\bigskip
	\begin{itemize}[leftmargin=*]
		\item Multiply the numerators and the denominators by each other.
		\item That is it!
	\end{itemize}
\end{definition}

In fact, we’ve already multiplied some fractions in this chapter (in Examples \ref{ArithmeticwithFractionsExample1} and \ref{ArithmeticwithFractionsExample2} we multiplied to get fractions into their common denominators).

\exam{\label{ArithmeticwithFractionsExample3}Evaluate $\displaystyle \frac{3}{5} \left( \frac{2}{7} \right)$
}
\indenttext{Using the above for multiplying fractions:

$$\displaystyle \frac{3}{5} \left( \frac{2}{7} \right) = \frac{3\cdot 2}{5 \cdot 7} = \frac{6}{35}$$
}

Dividing fractions is a little less straightforward.

%%%%%%%%%%%%%%%%%%%%%%%%%%%%%%%%%%%%%%%%%%%%%%%%%%%%%%%%%%%%%%%%%%%%%%
%
% Definition: Division of Fractions
%
%%%%%%%%%%%%%%%%%%%%%%%%%%%%%%%%%%%%%%%%%%%%%%%%%%%%%%%%%%%%%%%%%%%%%%

\begin{definition}
	\textbf{\underline{Dividing Fractions}}\\
	\bigskip
	\begin{itemize}[leftmargin=*]
		\item Write the division as a \textbf{complex fraction}, that is, as a fraction whose numerator and/or
		denominator are fractions.
		\item Multiply the complex fraction by the reciprocal of its denominator over itself.
		\item After multiplying, the denominator of the complex fraction becomes 1, and so it can be
		ignored. 
	\end{itemize}
\end{definition}

This method is also sometimes described as the \quotes{invert and multiply rule}, since the result ends up being equal to the numerator of the complex fraction multiplied by the reciprocal of the complex fraction’s denominator.

\exam{\label{ArithmeticwithFractionsExample4}Evaluate $\displaystyle \frac{3}{7} \div \frac{4}{5}$
}

\indenttext{Using the above for dividing fractions:

\begin{align*}
	\frac{3}{7} \div \frac{4}{5} & = \frac{\frac{3}{7}}{\frac{4}{5}} \\
	\\
	& =\frac{\frac{3}{7}}{\frac{4}{5}} \left( \frac{\frac{5}{4}}{\frac{5}{4}} \right) \\
	& = \frac{\frac{3\cdot 5}{7 \cdot 4}}{\frac{4 \cdot 5}{5\cdot 4}} \\
	\\
	& = \frac{\frac{15}{28}}{1} \\
	\\
	& = \frac{15}{28}
\end{align*}
}

\bigskip 

%%%%%%%%%%%%%%%%%%%%%%%%%%%%%%%%%%%%%%%%%%%%%%%%%%%%%%%%%%%%%%%%%%%%%%
%
% Subsection: Arithmetic with Fractions: Reducing Fractions to Lowest Terms
%
%%%%%%%%%%%%%%%%%%%%%%%%%%%%%%%%%%%%%%%%%%%%%%%%%%%%%%%%%%%%%%%%%%%%%%

\subsection{Reducing Fractions to Lowest Terms}

There are $60$ minutes in an hour, and so $30$ minutes is $\frac{30}{60}$ of an hour. But no one would actually say \quotes{$\frac{30}{60}$ of an hour}; of course you’d say $30$ minutes is $\frac{1}{2}$ of an hour. The fractions $\frac{30}{60}$ and $\frac{1}{2}$ are equal to each other, so either would be technically correct, but it’s obviously much simpler to use the fraction with the smaller numerator and denominator.

When we replace a fraction like $\frac{30}{60}$ to an equal fraction with a smaller numerator and denominator, we say that we \textbf{reduce}\index{Fractions!Reduced} it. When a fraction cannot be reduced any further, we say it is written in \textbf{lowest terms}. To keep things as simple as possible, it is a standard practice to reduce fractions to lowest terms whenever possible.

A fraction can be reduced if and only if its numerator and denominator share a common factor (other than $1$). For example, $\displaystyle \frac{30}{60}$ can be reduced because its numerator and denominator have common factors: $2$, $3$, $5$, $6$, $10$, $15$, and $30$. Since $30$ is the largest, we’ll use that, and rewrite $\displaystyle \frac{30}{60}$ as:

$$\frac{30}{60} = \frac{30 \times 1}{30 \times 2}$$

which, because of the rules for multiplying fractions, we can rewrite as:

$$\frac{30}{30}\left(\frac{1}{2}\right)$$

and since $\displaystyle \frac{30}{30} = 1$, we see that the result just is equal to $\displaystyle \frac{1}{2}$.

Usually we aren’t quite so formal about this; more often, we talk about canceling a common factor which amounts to the same thing. Following that route, once we see the common factor of $30$ in the numerator and denominator, we \quotes{cancel it out} like this:

$$\frac{30}{60} = \frac{30 \cdot 1}{30 \cdot 2} = \frac{\cancel{30} \cdot 1}{\cancel{30} \cdot 2} = \frac{1}{2}$$

It’s good to have some idea of why this works, but from a practical standpoint we usually just go with this cancellation approach.

%%%%%%%%%%%%%%%%%%%%%%%%%%%%%%%%%%%%%%%%%%%%%%%%%%%%%%%%%%%%%%%%%%%%%%
%
% Definition: Reducing Fractions
%
%%%%%%%%%%%%%%%%%%%%%%%%%%%%%%%%%%%%%%%%%%%%%%%%%%%%%%%%%%%%%%%%%%%%%%

\begin{definition}
	\textbf{\underline{Method for Reducing Fractions}}\\
	\bigskip
	\begin{itemize}[leftmargin=*]
		\item If the numerator and denominator share a common factor, cancel it.
		\item If the resulting fraction’s numerator and denominator don’t share any common factors, you’re done, the fraction has been reduced to lowest terms. If they do have a common factor, repeat.
	\end{itemize}
\end{definition}

Examples \ref{ArithmeticwithFractionsExample1} through \ref{ArithmeticwithFractionsExample4} were intentionally set up so that the final answer would come out to be in lowest terms. You cannot always assume that this will happen, though. Whenever you do any fraction arithmetic, you should as a last step reduce the answer to lowest terms.

\exam{\label{ArithmeticwithFractionsExample5}Evaluate $\displaystyle \frac{1}{6}+\frac{1}{3}-\frac{1}{8}$}

\indenttext{The least common denominator here is $24$. So:

$$\displaystyle \frac{1}{6}\left(\frac{4}{4}\right) + \frac{1}{3}\left(\frac{8}{8}\right)+ \frac{1}{8}\left(\frac{3}{3}\right)=\frac{4}{24} + \frac{8}{24}-\frac{3}{24}=\frac{9}{24}$$

Now we need to see if this fraction can be reduced to lower terms.  Since $9$ and $24$ share a common factor $3$ this should not be considered the final answer yet. We still need to reduce the fraction to lowest terms.

$$\frac{9}{24}=\frac{3 \cdot 3}{3 \cdot 8}=\frac{\cancel{3} \cdot 3}{\cancel{3} \cdot 8}=\frac{3}{8}$$
}

\bigskip

%%%%%%%%%%%%%%%%%%%%%%%%%%%%%%%%%%%%%%%%%%%%%%%%%%%%%%%%%%%%%%%%%%%%%%
%
% Subsection: Arithmetic with Fractions: Fraction Arithmetic and the Calculator
%
%%%%%%%%%%%%%%%%%%%%%%%%%%%%%%%%%%%%%%%%%%%%%%%%%%%%%%%%%%%%%%%%%%%%%%

\subsection{Fraction Arithmetic and Your Calculator}

Your calculator can be used as a tool for fraction arithmetic.

On any calculator, you can evaluate arithmetic expressions involving fractions. If you want to add $\displaystyle \frac{1}{2} + \frac{1}{4}$ you can simply enter that expression into your calculator and get a correct result.

Unfortunately, most calculators will give that answer as a decimal, not as a fraction.  So, for this example you would get an answer of $0.75$. Now, if a decimal answer is what you are looking for, that’s fine, but if you want it as a fraction, you’d need to recognize that $0.75$ is $\frac{3}{4}$ yourself. That’s fine for a familiar fraction like $\frac{3}{4}$, but it wouldn’t work so well if you were evaluating $\displaystyle \frac{2}{7} - \frac{6}{5}\left(\frac{2}{39}\right)$.  You’re not likely to recognize that your calculator’s decimal answer of $0.22417582$ is actually $\displaystyle \frac{306}{1365}$.

The TI-83 and TI-84 calculators have the ability, though, to report answers as fractions, if you tell them to.

On your calculator, locate the \quotes{MATH} key (it should be in the leftmost column, third or fourth from the top). If you hit that key it will take you to a menu, and the first item on that menu is \quotes{1: $\blacktriangleright$ Frac}.

This command tells the calculator to express the answer as a fraction in lowest terms.  So, for example, suppose we want to add $\displaystyle \frac{1}{2} + \frac{1}{4}$ and get the answer as a fraction, not a decimal. We would enter:

\quotes{$$1/2+1/4 \blacktriangleright \text{Frac}$$}

and then hit return to get the answer:

$$\frac{3}{4}$$

We can use this with any fractional expression we want to evaluate. Let’s revisit Example \ref{ArithmeticwithFractionsExample5} now:

\exam{\label{ArithmeticwithFractionsExample6}Evaluate $\displaystyle \frac{1}{6} + \frac{1}{3} - \frac{1}{8}$ using your calculator.}

\indenttext{We enter:

\quotes{$$1/6+1/3-1/8 \blacktriangleright \text{Frac}$$}

Hit enter and then get the result:

$$\frac{3}{8}$$
}

We can also use this feature to reduce a fraction to lowest terms.  Since the calculator always returns the result in lowest terms, if you enter a fraction that is not, it will evaluate it as a decimal and then convert the result to a fraction in lowest terms.  The end result - you’ve reduced your fraction.

\exam{\label{ArithmeticwithFractionsExample7}Reduce $\displaystyle \frac{5544}{22,176}$ to lowest terms}

\indenttext{On the calculator:

\quotes{$$5544/22176 \blacktriangleright \text{Frac}$$}

Gives the result:

$$\frac{1}{4}$$
}

\bigskip

%%%%%%%%%%%%%%%%%%%%%%%%%%%%%%%%%%%%%%%%%%%%%%%%%%%%%%%%%%%%%%%%%%%%%%
%
% Subsection: Arithmetic with Fractions: Closing Thoughts
%
%%%%%%%%%%%%%%%%%%%%%%%%%%%%%%%%%%%%%%%%%%%%%%%%%%%%%%%%%%%%%%%%%%%%%%

\subsection{Closing Thoughts}

With the \quotes{$\blacktriangleright$Frac} feature of the calculator available to us, we can easily handle otherwise tedious fraction arithmetic. This is really, really good news. One word of warning is very important here.  You may get the impression that, since the calculator can essentially do any fraction arithmetic you need, that you don’t really need to know the rules and techniques of fraction arithmetic.  While it is true that you’ll be able to avoid a lot of this work by effective calculator use, you will still need these rules and techniques for algebra purposes later in the course.  It is for this reason especially that homework exercises are provided where you are asked not to rely on the calculator for your fraction arithmetic. If you skip the opportunity to master those skills here, you’ll have much more trouble dealing with algebraic fractions later on!

%%%%%%%%%%%%%%%%%%%%%%%%%%%%%%%%%%%%%%%%%%%%%%%%%%%%%%%%%%%%%%%%%%%%%%
%
% Subsection: Arithmetic with Fractions: Exercises
%
%%%%%%%%%%%%%%%%%%%%%%%%%%%%%%%%%%%%%%%%%%%%%%%%%%%%%%%%%%%%%%%%%%%%%%

\clearpage

\subsection{Exercises}

Evaluate the following arithmetic expressions without using the \quotes{$\blacktriangleright$Frac} command on your calculator.

\begin{tasks}[label={}](2)
	\task \ex{$\frac{1}{2}+\frac{1}{3}$} \sol{$\frac{5}{6}$}
	\task \ex{$\frac{2}{7}+\frac{1}{5}$} 
	\task \ex{$\frac{3}{5}+\frac{1}{8}-\frac{1}{4}$} \sol{$\frac{19}{40}$}
	\task \ex{$\frac{6}{7}-\frac{3}{4}+\frac{3}{14}$} 
	\task \ex{$\frac{11}{8}-\frac{13}{16}$} \sol{$\frac{9}{16}$} 
	\task \ex{$\frac{11}{9}+\frac{2}{27}$} 
	\task \ex{$\frac{7}{30}+\frac{5}{6}-\frac{11}{12}$} \sol{$\frac{3}{20}$} 
	\task \ex{$\frac{13}{15}-\frac{2}{25}-\frac{4}{9}$} 
	\task \ex{$3-\frac{7}{5}-\frac{3}{10}$} \sol{$\frac{13}{10}$}
	\task \ex{$\frac{3}{8}-\frac{1}{12}+1$} 
	\task\ex{$\frac{4}{5}\left(\frac{1}{3}\right)$} \sol{$\frac{4}{15}$}
	\task\ex{$\frac{2}{7}\left(\frac{3}{5}\right)$}
	\task\ex{$\frac{1}{3}\left(\frac{5}{6}\right)$} \sol{$\frac{5}{18}$}
	\task\ex{$\frac{5}{3}\left(\frac{2}{9}\right)$}
	\task\ex{$\frac{3}{5}\div\frac{11}{2}$} \sol{$\frac{6}{55}$}
	\task\ex{$\frac{7}{2}\div\frac{5}{3}$}
	\task\ex{$\frac{9}{11}\div\frac{2}{3}$} \sol{$\frac{27}{22}$}
	\task\ex{$\frac{4}{5}\div\frac{3}{2}$}
	\task\ex{$\frac{7}{2}\div 5$} \sol{$\frac{7}{10}$}
	\task\ex{$\frac{7}{13}\div 2$}
	\task\ex{$\frac{3}{8}+\frac{3}{2}-\frac{5}{8}$} \sol{$\frac{5}{4}$}
	\task\ex{$\frac{5}{2}+\frac{1}{3}-\frac{5}{6}$}
	\task\ex{$\frac{4}{5}\left(\frac{25}{8}\right)$} \sol{$\frac{5}{2}$}
	\task\ex{$\frac{9}{16}\left(\frac{10}{3}\right)$}
	\task\ex{$\frac{10}{27}\div\frac{25}{9}$} \sol{$\frac{2}{15}$}
	\task\ex{$\frac{18}{35}\div\frac{9}{49}$}
\end{tasks}

\bigskip

Reduce the following fractions to lowest terms without using the \quotes{$\blacktriangleright$Frac} command on your calculator.

\begin{tasks}[label={}](2)
	\task\ex{$\frac{20}{60}$} \sol{$\frac{1}{3}$}
	\task\ex{$\frac{100}{80}$}
	\task\ex{$\frac{72}{140}$} \sol{$\frac{18}{35}$}
	\task\ex{$\frac{54}{72}$}
\end{tasks}

\bigskip

Evaluate the following arithmetic expressions, expressing your answers as fractions in lowest terms.  Use your calculator as much or as little as you like.

\begin{tasks}[label={}](2)
	\task\ex{$\frac{27}{4}\left(\frac{100}{81}\right)$} \sol{$\frac{25}{3}$}
	\task\ex{$\frac{162}{125}\left(\frac{200}{243}\right)$}
	\task\ex{$\frac{15}{48}-\frac{3}{8}+\frac{5}{6}$} \sol{$\frac{37}{48}$}
	\task\ex{$\frac{21}{25}-\frac{31}{125}+\frac{3}{10}$}
	\task\ex{$\frac{17}{75}-\frac{3}{10}+\frac{41}{250}$} \sol{$\frac{34}{375}$}
	\task\ex{$\frac{45}{132}-\frac{157}{430}+\frac{1}{4}$}
	\task\ex{$\frac{7}{4}-\frac{3}{8}+\frac{2}{5}\left(\frac{5}{16}\right)$} \sol{$\frac{3}{2}$}	
	\task\ex{$\frac{1}{2}-\frac{3}{32}+\frac{5}{96}\left(\frac{2}{35}\right)$}
	\task\ex{$\frac{24}{75}\left(\frac{27}{80}\right)$} \sol{$\frac{27}{250}$}
	\task\ex{$\frac{312}{49}\left(\frac{63}{78}\right)$}
	\task\ex{$\frac{824}{49}\div\frac{404}{32}$} \sol{$\frac{6592}{4949}$}
	\task\ex{$\frac{14}{29}\div\frac{105}{58}$}
	\task\ex{$\frac{5}{7}\left(\frac{14}{25}\right)$} \sol{$\frac{2}{5}$}
	\task\ex{$\frac{13}{4}-\frac{5}{2}+1$}
	\task\ex{$\frac{13}{8}-\frac{1}{4}$} \sol{$\frac{11}{8}$}
	\task\ex{$\frac{7}{2}\left(\frac{81}{21}\right)$}
	\task\ex{$\frac{9}{8}+\frac{3}{16}-2\left(\frac{5}{32}\right)$} \sol{$1$}
	\task\ex{$\frac{17}{32}\div\frac{54}{68}$}
	\task\ex{$\frac{6}{5}-\frac{3}{10}$} \sol{$\frac{9}{10}$}
	\task\ex{$\frac{243}{1000}+\frac{243}{750}$}
	\task\ex{$\frac{243}{1000}\div\frac{243}{750}$} \sol{$\frac{3}{4}$}
	\task\ex{$\frac{13}{15}-\frac{3}{25}\div\frac{5}{2}$}
	\task\ex{$3-\frac{7}{16}\div\frac{1}{8}$} \sol{$-\frac{1}{2}$}
	\task\ex{$\frac{3}{4}\left(\frac{45}{6}\right)\left(\frac{1}{9}\right)$}
\end{tasks}

\clearpage

%%%%%%%%%%%%%%%%%%%%%%%%%%%%%%%%%%%%%%%%%%%%%%%%%%%%%%%%%%%%%%%%%%%%%%
